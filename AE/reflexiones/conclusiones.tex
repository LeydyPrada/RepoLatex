\chapter{RESULTADOS Y DISCUSIÓN}

Los resultados obtenidos por el uso de la herramienta cumplen con las expectativas generadas, lo cual nos permite asegurar que el uso del prototipo para la planeación y ejecución de asambleas generales agiliza todo el proceso que esto embarca.

\vspace{0.5cm}

Aunque algunas personas, en especial las de mayor edad no presentaron una buena acogida a la utilización del dispositivo movil porque para ellos no generaba confianza el proceso, esto también se había observado en el levantamiento de información, los resultados analizados nos presentaban un porcentaje de entre el 8 al 13 porciento de personas que preferian seguir con el proceso tradicional.

\chapter{CONCLUSIONES}

A continuación se presentan las conclusiones generadas en el desarrollo del trabajo de grado.

\begin{itemize}
	\item La implementación del prototipo de la herramienta web facilita la planeación de las asambleas, agiliza la ejecución en cuanto al registro de asistentes y votación para la toma de decisiones, y por último, permite la consolidación de información y presentación de resultados en tiempo real durante la asamblea. Esto proporciona rapidez, fiabilidad y eficiencia en el manejo de información de las propiedades horizontales dando beneficios inherentes a los procesos.	
	\item El uso de un sistema en el desarrollo de los procesos de asambleas generales cobra importancia por el lado de soluciones con bases tecnológicas innovadoras, ya que la modularidad que presenta el sistema la utilización por diferentes usuarios al mismo tiempo no será inconveniente para la plataforma.	
	\item Con el desarrollo de este proyecto de grado se ha podido evidenciar de una manera muy significativa e importante que el paso por todo un proceso de profesionalización a punto de culminar nos ha permitido adquirir las habilidades y el conocimiento necesario para planear, diseñar e implementar una solución tecnológica integral.	
	\item Todo lo mencionando anteriormente enmarcado en un interés por mejorar las diferentes herramientas utilizadas actualmente en el mercado y ofrecer, como en el caso del curso de seminario de investigación desarrollos innovadores.
\end{itemize}

\section{Verificación, contraste y evaluación de los objetivos}

Para realizar la verficación de la investigación se realizan los sigueintes cuestionamientos que nos permiten determinar si la fuente de información seleccionada fue la adecuada para el problema planteado.

\begin{enumerate}
	\item ¿Hay fuentes de datos ya existentes y explotables?
	\item[-] Para la problematica planteada no se encontraron datos existentes que pretendierán solucionar lo trabajado en el proyecto de investigación.
	
	\item ¿Cuán fiables son estas fuentes de datos?
	\item[-] Las fuentes de datos son 100 porciento fiables, se trabajo con personas que habitan en propiedad horizontal y asisten a las asambleas de copropietarios.
	
	\item ¿Se puede fácilmente adaptar estas fuentes de datos según las necesidades del
	proyecto o realizar análisis secundarios?
	\item[-] Las fuentes de datos son especificas para el proyecto lo cual facilita el análisis de los datos.
	
	\item ¿Es necesario recopilar datos adicionales y/o iniciar estudios? En este caso se debe tomar en cuenta la disponibilidad de recursos necesarios: Personal calificado, Recursos financieros, Tiempo, etc.
	\item[-] La información recopilada satisface el levantamiento de información para poder evaluar el desarrollo de la herramienta, por lo anterior no se requiere recopilar información adicional o iniciar estudios.
\end{enumerate}

\section{Síntesis del modelo propuesto}

La solución dada por el modelo está sustentada en los procesos desarrollados en las asambleas generales, por ello es importante tener en cuenta que ésta debe ser realizada por personal que conozca el flujo de los procesos para el desarrollo de las asambleas.

\vspace{0.5cm}

El desarrollo del prototipo contempla un sistema web que permita ejecutar los procesos la planeación (identificación de objetivos, los puntos del orden del día, resumir y revisar las asignaciones del día), ejecución (registro y votación de los temas planificados) y recopilación de información (consolidación y presentación de información recopilada mediante conteo digital y generación de informes) de las asambleas generales de copropietarios de propiedad horizontal mediante el uso de servicios SOAP. 


\section{Aportes originales}

Las aportaciones originales de este trabajo de grado, directamente relacionadas con las conclusiones anteriores, son las siguientes: 

\begin{itemize}
	\item Creación del módulo de planeación que le permite al administrador crear las asambleas de manera mas eficiente.
	\item Creación del módulo de ejecución de asambleas generales lo cual facilita y agiliza las mismas.
	\item Creación del módulo de presentación de resultados.
\end{itemize}

Aunque actualmente existe una empresa que presta sus servicios para la ejecución de asambleas generales, en el mercado de aplicaciones no existe una herramienta que le permita al administrador tener el control desde la planeación hasta la consolidación de resultados de asambleas generales. Este es un enfoque diferente al convencional porque se acostumbra a realizar las asambleas por medio de los procesos tradicionales y dadas las nuevas propiedades horizontales que estan en crecimiento se requieren herramientas que faciliten sus procesos. 

\vspace{0.5cm}

El aporte generado en esta investigación demuestra el interes de copropietarios de propiedad horizontal en adquirir nuevas herramientas.

\section{Trabajos o Publicaciones derivadas}

Actualmente no se cuenta con trabajos o publicaciones derivadas de este trabajo investigativo.

\newpage


\chapter{PROSPECTIVA DEL TRABAJO DE GRADO}

\section{Líneas de investigación futuras}

En lo que concierne a las líneas de investigación futura, durante el proceso de
elaboración de este trabajo se han considerado interesantes los temas que se exponen a
continuación. 

\vspace{0.5cm}

En primer lugar, la ampliación del mercado evaluado, verificando si el software a desarrollar podría ser implementado en diferentes ambitos de reunión que permitan sistematizar el proceso y reducir los tiempos de ejecución que estas conllevan.

\vspace{0.5cm}

Por otra parte es importante resaltar que en el sector de la investigación (Propiedad horizontal) no se encuentra información referente a estudios previos que ayuden a generar un estudio mas detallado de los escenarios que se presentan al interior de estas propiedades y con ellos poder generar un alcance mayor a las necesidades que presenta este sector.

\vspace{0.5cm}

Finalmente, el ambito de las propiedades horizontales nos permite explorar un nuevo mundo de investigación que puede además, centrarse en el comportamiento de las personas cuando son citadas a asambleas generales tediosas y demoradas, con eso poder determinar diferentes planes de mitigación que permitan una exitosa ejecución de las asambleas generales. 

\vspace{0.5cm}

\section{Trabajos de Investigación futuros}

Dentro de los trabajos futuros de investigación se pretende continuar con el desarrollo del prototipo a beta para poder evaluar de mejor manera de realizar las asambleas generales en propiedad horizontal, adicionalmente esto también genera la ampliación del espectro que se tenía acerca de las herramientas utilizadas en la ejecución de las mismas. 

\vspace{0.5cm}

Una vía que podría ser alterna al desarrollo completo del aplicativo es el estudio de las conductas generadas después de la participación en una asamblea general. Poder evaluar el estado de ánimo, el grado de satisfacción, la retrospectiva y las expectativas que tiene el copropietario.



\newpage