\section{Análisis de los resultados}

\subsubsection{Análisis de los resultados}

Como se presenta en el apartado anterior la información presentá la inconformidad vivida en un grupo especifico de propiedades horizontales, para el estudio de investigación esta población fue tomada de los conjuntos residenciales Parque central Bonavista 1, Parque central Bonavista 2 y Torres de Bellavista ubicados en el barrio el perdomo en el sur de la ciudad de Bogotá. La caracteristica principal para centrarse en este grupo de conjuntos residenciales fue la constructura que desarrollo los proyectos arquitectonicos, la cual fue AR Construcciones.

\vspace{0.5cm}

La distribución poblacional de cada uno de los conjuntos se encuentra de la siguiente manera:

\begin{itemize}
	\item Parque central Bonavista 1 = 1080 Apartamentos
	
	\item Parque central Bonavista 2 = 1296 Apartamentos
	
	\item Torres de Bellavista = 1500 Apartamentos
\end{itemize}

Cada propiedad horizontal supera las 1000 unidades de apartamentos, cada una de estas propiedades por ley requiere realizar una asamblea general de propietarios, estas pueden llegar a durar mas de 7 u 8 horas como se plantea en la definición del problema en este documento.

Al realizar la encuesta en un numero limitado de personas que conviven en estas propiedades horizontales se observa que ellos manifiestan la necesidad de adquirir herramientas que agilicen y faciliten la ejecución de dichas asambleas. 

Como se observa en los datos de la información recopilada en cada una de las preguntas más del 90 pociento se encuentran inconformes con la manera actual de ejecución de asambleas generales.

\subsubsection{Verificación de preguntas de investigación}

La verificación de las preguntas de investigación generadas al inicio de este proceso se presentan a continuación:

\begin{itemize}
	\item ¿De qué forma asegurar que los temas que deben ser tratados en una asamblea general de copropietarios de propiedad horizontal sean los de su exclusiva competencia y dejar los temas generales para cada órgano competente?
	
	\item[RTA:] Dentro del desarrollo se plantea el uso de la planificación de asambleas que permite a cada copropietario la lectura oportuna de los termas a tratar en cada asamblea, con esto permitir en caso de duda acudir con antelación a la administración para resolver las inquitudes y no esperar hasta la ejecución de la asamblea.
	
	\item ¿Cómo agilizar y regular el proceso de votación durante la toma de decisiones en la ejecución de la asamblea general de copropietarios de propiedad horizontal?
	
	\item[RTA:] La votación para cada una de las decisiones que deban tomarse dentro de la asamblea general de copropietarios se realizará por medio de la aplicación web lo cual facilitará la recolección de votos y su conteo.
	
	\item ¿Cómo generar de forma rápida y eficiente el informe de resultados de la asamblea general de copropietarios para la socialización y evaluación de la información?
	
	\item[RTA:] Al tener un mayor control de la votación gracias a lo planteado en el item anterior, la presentación de resultados se facilitará al contar con la información sistematizada en poco tiempo, con esto los copropietarios podrán conocer los resultados a cada item tratado dentro de la asamblea.
	
\end{itemize}

\subsubsection{Verificación de hipótesis}

\textbf{\textit{Hipótesis del trabajo}}

\vspace{0.5cm}

Si se proporciona a las propiedades horizontales un prototipo de aplicación web que permita el desarrollo de las asambleas generales de copropietarios de manera ordenada, ágil, sencilla y eficaz, se aumentará el control y disminuirá considerablemente el tiempo en el proceso de toma de decisiones, se reducirán costos de logística, permitirá un mejor manejo de la información y generará un uso eficiente de las tecnologías de información.

\vspace{1cm}

VERIFICACIÓN:

Para la verificación de la hipótesis se realiza una prueba de concepto con algunos usuarios pertenecientes a un conjunto residencial, los cvuales mostraron su grado ante la herramienta y pudieron validar que el tiempo de planeación y ejecución de la asamblea era disminuido considerablemente.

\newpage