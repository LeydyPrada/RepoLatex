
\section{INTRODUCCION}

\vspace{5mm}

Las asambleas generales de copropietarios de propiedad horizontal, son un dolor de cabeza constante para sus organizadores y sus participantes, las tareas de planeación y desarrollo de las mismas son un proceso largo en el cual debe ser tenida en cuenta cada una de las opiniones de los copropietarios dado que cada uno tiene parte en lo que compete a las decisiones de lo que se debe realizar en la propiedad.

\vspace{5mm}

Para tratar de suplir las necesidades generadas en las asambleas se plantea la creación de una herramienta web que permita planear, ejecutar y generar resultados para su presentación durante las asambleas, Entregando los resultados a los participantes en tiempo real.

\vspace{5mm}

El presente documento integra en su inicio los antecedentes para la definición y planeación del proyecto, la problemática y la base teórica y conceptual requeridos para la culminación de la primera fase del proyecto. Seguido a esto, se encuentran las fases de requerimientos, análisis, diseño, implementación y pruebas, que enmarcan las tres metodologías usadas y fusionadas en el desarrollo de este proyecto tales como: RUP para el desarrollo del sistema de software y la arquitectura orientada a servicios SOA.
\newpage